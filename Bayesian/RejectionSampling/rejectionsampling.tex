% Options for packages loaded elsewhere
\PassOptionsToPackage{unicode}{hyperref}
\PassOptionsToPackage{hyphens}{url}
%
\documentclass[
]{article}
\usepackage{amsmath,amssymb}
\usepackage{lmodern}
\usepackage{ifxetex,ifluatex}
\ifnum 0\ifxetex 1\fi\ifluatex 1\fi=0 % if pdftex
  \usepackage[T1]{fontenc}
  \usepackage[utf8]{inputenc}
  \usepackage{textcomp} % provide euro and other symbols
\else % if luatex or xetex
  \usepackage{unicode-math}
  \defaultfontfeatures{Scale=MatchLowercase}
  \defaultfontfeatures[\rmfamily]{Ligatures=TeX,Scale=1}
\fi
% Use upquote if available, for straight quotes in verbatim environments
\IfFileExists{upquote.sty}{\usepackage{upquote}}{}
\IfFileExists{microtype.sty}{% use microtype if available
  \usepackage[]{microtype}
  \UseMicrotypeSet[protrusion]{basicmath} % disable protrusion for tt fonts
}{}
\makeatletter
\@ifundefined{KOMAClassName}{% if non-KOMA class
  \IfFileExists{parskip.sty}{%
    \usepackage{parskip}
  }{% else
    \setlength{\parindent}{0pt}
    \setlength{\parskip}{6pt plus 2pt minus 1pt}}
}{% if KOMA class
  \KOMAoptions{parskip=half}}
\makeatother
\usepackage{xcolor}
\IfFileExists{xurl.sty}{\usepackage{xurl}}{} % add URL line breaks if available
\IfFileExists{bookmark.sty}{\usepackage{bookmark}}{\usepackage{hyperref}}
\hypersetup{
  pdftitle={Computing Integrals Monte Carlo Way},
  pdfauthor={Rahul Goswami},
  hidelinks,
  pdfcreator={LaTeX via pandoc}}
\urlstyle{same} % disable monospaced font for URLs
\usepackage[margin=1in]{geometry}
\usepackage{color}
\usepackage{fancyvrb}
\newcommand{\VerbBar}{|}
\newcommand{\VERB}{\Verb[commandchars=\\\{\}]}
\DefineVerbatimEnvironment{Highlighting}{Verbatim}{commandchars=\\\{\}}
% Add ',fontsize=\small' for more characters per line
\usepackage{framed}
\definecolor{shadecolor}{RGB}{248,248,248}
\newenvironment{Shaded}{\begin{snugshade}}{\end{snugshade}}
\newcommand{\AlertTok}[1]{\textcolor[rgb]{0.94,0.16,0.16}{#1}}
\newcommand{\AnnotationTok}[1]{\textcolor[rgb]{0.56,0.35,0.01}{\textbf{\textit{#1}}}}
\newcommand{\AttributeTok}[1]{\textcolor[rgb]{0.77,0.63,0.00}{#1}}
\newcommand{\BaseNTok}[1]{\textcolor[rgb]{0.00,0.00,0.81}{#1}}
\newcommand{\BuiltInTok}[1]{#1}
\newcommand{\CharTok}[1]{\textcolor[rgb]{0.31,0.60,0.02}{#1}}
\newcommand{\CommentTok}[1]{\textcolor[rgb]{0.56,0.35,0.01}{\textit{#1}}}
\newcommand{\CommentVarTok}[1]{\textcolor[rgb]{0.56,0.35,0.01}{\textbf{\textit{#1}}}}
\newcommand{\ConstantTok}[1]{\textcolor[rgb]{0.00,0.00,0.00}{#1}}
\newcommand{\ControlFlowTok}[1]{\textcolor[rgb]{0.13,0.29,0.53}{\textbf{#1}}}
\newcommand{\DataTypeTok}[1]{\textcolor[rgb]{0.13,0.29,0.53}{#1}}
\newcommand{\DecValTok}[1]{\textcolor[rgb]{0.00,0.00,0.81}{#1}}
\newcommand{\DocumentationTok}[1]{\textcolor[rgb]{0.56,0.35,0.01}{\textbf{\textit{#1}}}}
\newcommand{\ErrorTok}[1]{\textcolor[rgb]{0.64,0.00,0.00}{\textbf{#1}}}
\newcommand{\ExtensionTok}[1]{#1}
\newcommand{\FloatTok}[1]{\textcolor[rgb]{0.00,0.00,0.81}{#1}}
\newcommand{\FunctionTok}[1]{\textcolor[rgb]{0.00,0.00,0.00}{#1}}
\newcommand{\ImportTok}[1]{#1}
\newcommand{\InformationTok}[1]{\textcolor[rgb]{0.56,0.35,0.01}{\textbf{\textit{#1}}}}
\newcommand{\KeywordTok}[1]{\textcolor[rgb]{0.13,0.29,0.53}{\textbf{#1}}}
\newcommand{\NormalTok}[1]{#1}
\newcommand{\OperatorTok}[1]{\textcolor[rgb]{0.81,0.36,0.00}{\textbf{#1}}}
\newcommand{\OtherTok}[1]{\textcolor[rgb]{0.56,0.35,0.01}{#1}}
\newcommand{\PreprocessorTok}[1]{\textcolor[rgb]{0.56,0.35,0.01}{\textit{#1}}}
\newcommand{\RegionMarkerTok}[1]{#1}
\newcommand{\SpecialCharTok}[1]{\textcolor[rgb]{0.00,0.00,0.00}{#1}}
\newcommand{\SpecialStringTok}[1]{\textcolor[rgb]{0.31,0.60,0.02}{#1}}
\newcommand{\StringTok}[1]{\textcolor[rgb]{0.31,0.60,0.02}{#1}}
\newcommand{\VariableTok}[1]{\textcolor[rgb]{0.00,0.00,0.00}{#1}}
\newcommand{\VerbatimStringTok}[1]{\textcolor[rgb]{0.31,0.60,0.02}{#1}}
\newcommand{\WarningTok}[1]{\textcolor[rgb]{0.56,0.35,0.01}{\textbf{\textit{#1}}}}
\usepackage{graphicx}
\makeatletter
\def\maxwidth{\ifdim\Gin@nat@width>\linewidth\linewidth\else\Gin@nat@width\fi}
\def\maxheight{\ifdim\Gin@nat@height>\textheight\textheight\else\Gin@nat@height\fi}
\makeatother
% Scale images if necessary, so that they will not overflow the page
% margins by default, and it is still possible to overwrite the defaults
% using explicit options in \includegraphics[width, height, ...]{}
\setkeys{Gin}{width=\maxwidth,height=\maxheight,keepaspectratio}
% Set default figure placement to htbp
\makeatletter
\def\fps@figure{htbp}
\makeatother
\setlength{\emergencystretch}{3em} % prevent overfull lines
\providecommand{\tightlist}{%
  \setlength{\itemsep}{0pt}\setlength{\parskip}{0pt}}
\setcounter{secnumdepth}{-\maxdimen} % remove section numbering
\ifluatex
  \usepackage{selnolig}  % disable illegal ligatures
\fi

\title{Computing Integrals Monte Carlo Way}
\author{Rahul Goswami}
\date{18/01/2022}

\begin{document}
\maketitle

\hypertarget{rejection-sampling-method}{%
\subsubsection{Rejection Sampling
Method}\label{rejection-sampling-method}}

Its always not easy to withdraw samples from posterior \(\pi(\theta|x)\)
distribution.Most of the time are not familiar with the functional form
of the posterior distribution.

Supoose we wanna take a sample from the posterior distribution
\(\pi(\theta|x)\) Then we will find another probability distribution
\(p(\theta)\) which have the following properties

\begin{enumerate}
\def\labelenumi{\arabic{enumi}.}
\tightlist
\item
  Easy to withdraw samples from
\item
  Resembles the posterior distribution
\item
  For all parameter \(\theta\) and a constant \(k\) ,
  \(\pi(\theta|x) \leq k p(\theta)\)
\end{enumerate}

\hypertarget{algorithm}{%
\subsubsection{Algorithm}\label{algorithm}}

\begin{enumerate}
\def\labelenumi{\arabic{enumi}.}
\tightlist
\item
  Take a sample from the from the distribution \(p(\theta)\) and a
  Uniform Random Variable \(U\)
\item
  If \(U < \frac{\pi(\theta|x)}{k \cdot p(\theta)}\) then accept the
  sample
\item
  If \(U > \frac{\pi(\theta|x)}{k \cdot p(\theta)}\) then reject the
  sample
\end{enumerate}

The Performance of the Rejection Sampling Method is measured by
Acceptance Rate.

\hypertarget{example}{%
\subsubsection{Example}\label{example}}

Suppose we want to withdraw samples from normal distribution with mean
\(\mu\) and variance \(\sigma\),which equivalent to get samples from
standard Normal distribution, because we just have to do a simple linear
transformation to get a distribution with mean \(\mu\) and variance
\(\sigma\).So we will be using the standard Normal distribution

Now we are taking proposaldensity or in some literature mentioned as
candidate density \(p(\theta)\) as an exponential distribution with mean
\(1\), while we know that exponential random variable is always positive
and hence we will be taking the absolute value of the Standard Normal
random variable, and then multiply iy by -1 by generating a uniform
random variable \(U\).Whenever \(U\) is less than 0.5

\[
p(\theta) = e^{-\theta}  \\
\pi(\theta|x) = \frac{2}{2\pi}  e^{-\frac{1}{2}(\theta)^2}1_{x \geq0}
\]

Then

\(\frac{\pi(\theta|x)}{p(\theta)}\) is the ratio of the posterior
distribution and the candidate density.It will be at maximum at
\(\theta = 1\) thus k = \(\sqrt{2e / \pi} \approx 1.32\)

Then steps for generating samples from the posterior distribution are as
follows 1. Take a sample from exponential distribution with mean \(1\)
and a uniform random variable U 2. If
\(U \leq \frac{\frac{1}{\sqrt{2\pi}}e^{-\frac{\theta^2}{2}}}{\sqrt{2e/\pi} e^{-\theta}} \ i.e \ U \leq e^{-(1 -\theta)^2/2}\)
then accept the sample 3. Generate another uniform random variable
\(U\), if U is less than 0.5 then multiply the sample by -1

\begin{Shaded}
\begin{Highlighting}[]
\NormalTok{nsample }\OtherTok{=} \DecValTok{10000}
\NormalTok{sample }\OtherTok{=} \FunctionTok{c}\NormalTok{()}
\NormalTok{count }\OtherTok{=} \DecValTok{0}
\ControlFlowTok{while}\NormalTok{(}\FunctionTok{length}\NormalTok{(sample) }\SpecialCharTok{\textless{}}\NormalTok{ nsample)\{}
\NormalTok{  U }\OtherTok{=} \FunctionTok{runif}\NormalTok{(}\DecValTok{1}\NormalTok{)}
\NormalTok{  count }\OtherTok{=}\NormalTok{ count }\SpecialCharTok{+} \DecValTok{1}
\NormalTok{  theta }\OtherTok{=} \FunctionTok{rexp}\NormalTok{(}\DecValTok{1}\NormalTok{)}
\NormalTok{  U2 }\OtherTok{=} \FunctionTok{runif}\NormalTok{(}\DecValTok{1}\NormalTok{)}
  \ControlFlowTok{if}\NormalTok{(U }\SpecialCharTok{\textless{}=} \FunctionTok{exp}\NormalTok{((}\SpecialCharTok{{-}}\NormalTok{(}\DecValTok{1}\SpecialCharTok{{-}}\NormalTok{theta)}\SpecialCharTok{\^{}}\DecValTok{2}\NormalTok{)}\SpecialCharTok{/}\DecValTok{2}\NormalTok{))\{ }
    \ControlFlowTok{if}\NormalTok{(U2 }\SpecialCharTok{\textless{}=} \FloatTok{0.5}\NormalTok{)\{}
\NormalTok{      theta }\OtherTok{=} \SpecialCharTok{{-}}\NormalTok{theta}
\NormalTok{    \}}
    
\NormalTok{    sample }\OtherTok{=} \FunctionTok{c}\NormalTok{(sample, theta)}
\NormalTok{  \}}
\NormalTok{\}}
\FunctionTok{cat}\NormalTok{(}\StringTok{"Acceptance Rate: "}\NormalTok{, count}\SpecialCharTok{/}\NormalTok{nsample)}
\end{Highlighting}
\end{Shaded}

\begin{verbatim}
## Acceptance Rate:  1.3135
\end{verbatim}

\begin{Shaded}
\begin{Highlighting}[]
\FunctionTok{plot}\NormalTok{(}\FunctionTok{density}\NormalTok{(sample))}
\end{Highlighting}
\end{Shaded}

\includegraphics{rejectionsampling_files/figure-latex/RejectionSampling-1.pdf}

\end{document}
